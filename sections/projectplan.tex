\chapter {Project Plan}
% \chapter{Evaluation}
\section{Work Breakdown Structure (WBS)}
To ensure smooth progress and timely delivery, the development of the MStacking framework is organized into seven structured phases. Each phase includes specific tasks that align with the overall project objectives:

\begin{table}[htbp]
\centering
\caption{Project Workflow Phases}
\label{tab:workflow}
\renewcommand{\arraystretch}{1.5} % Adjusted row spacing
\setlength{\tabcolsep}{6pt} % Increased horizontal padding
\begin{tabular}{|>{\bfseries}l!{\vrule width 1.2pt}p{9.5cm}|}
\hline
\rowcolor{gray!15}
\textbf{Phase} & \textbf{Description} \\ \hline

1. Research \& Literature Review & 
Study recent work in federated learning, stacking, heart sound classification, and AI in healthcare. \\ \hline

2. System Design & 
\vspace{-4mm}\begin{itemize}[leftmargin=*,nosep]
\item Define the MStacking architecture.
\item Plan client-server interactions and simulation topology.
\item Outline metadata and communication protocol.
\end{itemize} \\ \hline

3. Data Preparation & 
\vspace{-4mm}\begin{itemize}[leftmargin=*,nosep]
\item Preprocess PCG signals using MFCC and CWT.
\item Partition datasets into star-structured splits for FL simulation.
\item Validate multimodal input formats.
\end{itemize} \\ \hline

4. Local Model Implementation & 
\vspace{-4mm}\begin{itemize}[leftmargin=*,nosep]
\item Implement and train Random Forest, FNN, and CNN models.
\item Optimize each for local client data characteristics.
\end{itemize} \\ \hline

5. Stacking \& Federated Aggregation & 
\vspace{-4mm}\begin{itemize}[leftmargin=*,nosep]
\item Design metadata schema (predictions, densities).
\item Build meta-learner.
\item Integrate client outputs into a unified global model.
\end{itemize} \\ \hline

6. Evaluation \& Validation & 
\vspace{-4mm}\begin{itemize}[leftmargin=*,nosep]
\item Run experiments under non-IID, imbalanced, and noisy data scenarios.
\item Compare with centralized and traditional FL models.
\item Use metrics like accuracy, F1-score, AUROC, SHAP.
\end{itemize} \\ \hline

7. Documentation \& Presentation & 
\vspace{-4mm}\begin{itemize}[leftmargin=*,nosep]
\item Prepare final report, figures, and codebase documentation.
\item Present results and insights clearly for academic submission.
\end{itemize} \\ \hline
\end{tabular}
\end{table}

\begin{itemize}[leftmargin=*]
\item \textbf{Timeline:} Each phase is allocated 2-3 weeks with overlapping activities where possible
\item \textbf{Dependencies:} Sequential flow with iterative refinement loops between phases
\item \textbf{Deliverables:} Technical reports, prototype implementations, and evaluation results at each phase
\end{itemize}


\section{Timeline and Gantt Chart}
The project will be completed over a  four-month period, with major deliverables planned at each stage. Below  is the proposed timeline in a Gantt chart format:

\begin{figure}[htbp]
\centering
\begin{ganttchart}[
    % Chart configuration
    x unit=0.8cm,
    y unit title=0.8cm,
    y unit chart=0.8cm,
    vgrid,
    hgrid,
    title height=1,
    title label font=\bfseries\footnotesize,
    bar/.append style={fill=blue!30, rounded corners=2pt},
    bar height=0.6,
    bar label font=\footnotesize,
    milestone/.append style={fill=red!50, rounded corners=2pt},
    milestone label font=\footnotesize,
    group/.append style={draw=black, fill=green!20},
    group left shift=0,
    group right shift=0,
    group height=.3,
    group peaks tip position=0,
    bar incomplete/.append style={fill=yellow!30}
]{1}{16} % 16 weeks total
    
    % Months
    \gantttitle{July}{4}
    \gantttitle{August}{4}
    \gantttitle{September}{4}
    \gantttitle{October}{4} \\
    
    % Week numbers
    \gantttitlelist{1,...,16}{1} \\
    
    % Tasks
    \ganttbar{Literature Review \& Planning}{1}{5} \\
    \ganttbar{System Design}{1}{7} \\
    \ganttbar{Dataset Preparation}{5}{6} \\
    \ganttbar{Local Model Development}{5}{6} \\
    \ganttbar{Stacking \& Aggregation Module}{7}{12} \\
    \ganttbar{Evaluation \& Testing}{12}{14} \\
    \ganttbar{Documentation \& Final Report}{13}{16} \\
    \ganttbar{Presentation Preparation}{15}{16}
    
    % Vertical lines separating months
    \ganttvrule{4}{4}
    \ganttvrule{8}{8}
    \ganttvrule{12}{12}
\end{ganttchart}
\caption{Project Timeline Gantt Chart}
\label{fig:gantt}
\end{figure}

\section{Resources and Tools}
To develop, test,  and evaluate the MStacking framework efficiently, the following tools and platforms will be utilized:

\begin{itemize}[leftmargin=*]
\item \textbf{Software Tools:} Python 3.x, PyTorch, OpenSMILE, LibROSA, Flower, FedML, and Jupyter  Notebooks.
\item \textbf{Hardware:} University-provided GPU-enabled machines and cloud services like Google Colab and Kaggle  for additional compute support.
\item \textbf{Version Control:} Git and GitHub for collaborative development and version tracking.
\item \textbf{Documentation} Overleaf (LaTeX) for professional academic writing and formatting.
\item \textbf{Visualization Monitoring:} Matplotlib, Seaborn, and TensorBoard for experiment tracking and performance visualization.
\end{itemize}




\section{Risk Management}
Identified risks and corresponding mitigation strategies are outlined in the table below:
\begin{table}[htbp]
\centering
\caption{Risk and Mitigation Strategy}
\label{tab:workflow}
\renewcommand{\arraystretch}{1.3} % Adjusted row spacing
\setlength{\tabcolsep}{6pt} % Increased horizontal padding
\begin{tabular}{|<{\bfseries}l!{\vrule width 0.8pt}p{7.5cm}|}
\hline
\rowcolor{gray!15}
\textbf{Risk} & \textbf{Mitigation Strategy} \\ \hline

1. Insufficient compute resources & 
Leverage cloud-based platforms (Google Colab, Kaggle) for GPU access. \\ \hline

2. Time constraints during academic semester & 
Apply agile development cycles; prioritize the core MStacking framework before add-ons. \\ \hline

3. Lack of access to real-time medical data & 
Use realistic, publicly available datasets like PhysioNet to simulate clinical settings. \\ \hline

4. Debugging federated heterogeneity issues & 
Conduct isolated tests for each local model type (RF, FNN, CNN) before integration. \\ \hline


\end{tabular}
\end{table}

