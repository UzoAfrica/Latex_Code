\chapter{Evaluation}


\section{Evaluation Objectives}
\label{sec:evaluation_objectives}

The evaluation phase aims to rigorously assess how well the proposed MStacking framework performs under realistic, decentralized healthcare scenarios. Key goals include:

\begin{itemize}
    \item Measuring the classification accuracy of the global multi-class model built via stacking.
    \item Comparing MStacking’s performance to traditional centralized learning and standard FL methods like FedAvg.
    \item Testing robustness against client heterogeneity, label misalignment, and imbalanced data distributions.
    \item Assessing system efficiency by tracking communication overhead and computational cost.
    \item Exploring the interpretability of predictions using feature attribution tools such as SHAP.
\end{itemize}

\section{Evaluation Metrics}
\label{sec:evaluation_metrics}

To ensure a comprehensive analysis, the following metrics will be employed:

\begin{itemize}
    \item \textbf{Accuracy}: Overall rate of correct predictions.
    \item \textbf{Precision, Recall, F1-Score}: Evaluated per class to reflect detection sensitivity and specificity. Macro-averaged versions will be emphasized to address class imbalance.
    \item \textbf{AUROC}: Measures the model's ability to distinguish between classes under varying thresholds.
    \item \textbf{Confusion Matrix}: Highlights where and how often misclassifications occur.
    \item \textbf{Model Communication Cost}: Total size of metadata exchanged between clients and the server.
    \item \textbf{Training Time per Round}: Gauges the runtime efficiency of global model updates.
    \item \textbf{Model Interpretability}: Investigated via SHAP or similar techniques to understand decision rationale.
\end{itemize}

\section{Experimental Setup}
\label{sec:experimental_setup}

\textbf{Datasets:}
\begin{itemize}
    \item \textit{PhysioNet/CinC Challenge 2016}~\cite{clifford2016classification}: Main benchmark dataset, with 3,000+ annotated heart sound recordings.
    \item \textit{Additional PCG datasets}~\cite{liu2016open}: Used to simulate diverse client data conditions and class variety.
\end{itemize}

\textbf{Client Configuration:}
\begin{itemize}
    \item 5 to 10 virtual clients will be created.
    \item Each client will have a non-IID data partition containing the normal class and a distinct abnormal class (star-structure).
    \item Clients will use varying model types (e.g., RF, FNN, CNN) to simulate hardware and algorithmic diversity.
\end{itemize}

\textbf{Baselines for Comparison:}
\begin{itemize}
    \item Centralized CNN trained on full data.
    \item Traditional FL (FedAvg) with uniform models and aligned labels.
    \item Personalized FL using clustered or fine-tuned models.
    \item Ablation Study to evaluate the role of metadata: predictions only vs. predictions + densities.
\end{itemize}

\section{Evaluation Procedure}
\label{sec:evaluation_procedure}

\textbf{1. Training Phase:}
\begin{itemize}
    \item Clients train local binary classifiers.
    \item Output includes prediction probabilities and Gaussian Mixture-based density estimates.
    \item These are sent to the server and integrated into a global model using stacking.
\end{itemize}

\textbf{2. Validation Phase:}
\begin{itemize}
    \item A held-out multi-class dataset is used to evaluate the global model.
    \item Metrics are updated after each training round.
\end{itemize}

\textbf{3. Stress Testing:}
\begin{itemize}
    \item Client dropouts to test resilience.
    \item Label noise to assess robustness.
    \item Imbalanced class samples to evaluate fairness.
\end{itemize}

\textbf{4. Result Analysis:}
\begin{itemize}
    \item Results will be visualized using ROC curves, training/validation loss plots, and confusion matrices.
    \item Comparative tables will summarize performance against baselines.
    \item A detailed discussion will highlight the strengths and trade-offs of MStacking.
\end{itemize}

\begin{table}[h]
\centering
\caption{Performance Comparison with Existing Methods}
\label{tab:performance_comparison}
\renewcommand{\arraystretch}{1.2}
\begin{tabular}{|p{2cm}|p{1cm}|p{1.5cm}|p{1.5cm}|p{1.5cm}|p{1.5cm}|p{2cm}|p{2cm}|}
\hline
\textbf{Method} & \textbf{Model Type} & \textbf{Setting} & \textbf{Data Mode} & \textbf{Accuracy} & \textbf{UF1 Score} & \textbf{UAR Score} & \textbf{Notes} \\ \hline
Centralized CNN & CNN & Centralized & Balanced & Moderate & Moderate & Moderate & Traditional learning with full label sharing \\ \hline
Fed-BIDS~\cite{fedbids2022} & RF + NN Meta & Federated & Balanced & High & Moderate & Moderate & Blends metadata for intrusion detection \\ \hline
Fed-MStacking (Homog.) & RF / FNN / CNN & Federated & Balanced & Higher & Higher & Higher & Uses identical architectures across clients \\ \hline
Fed-MStacking (Hetero) & RF + FNN + CNN & Federated & Balanced & Highest & Highest & Highest & Supports diverse models \& misaligned class labels (This work) \\ \hline
\end{tabular}
\end{table}
